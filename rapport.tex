\documentclass{article}

\usepackage[utf8]{inputenc}
\usepackage{amsfonts}            %For \leadsto
\usepackage{amsmath}             %For \text

\title{Travail pratique \#1}
\author{Georgiy Gegiya, Cordeleanu Corneliu}

\begin{document}

\maketitle

\newcommand \mML {\ensuremath\mu\textsl{ML}}
\newcommand \kw [1] {\textsf{#1}}
\newcommand \id [1] {\textsl{#1}}
\newcommand \punc [1] {\kw{`#1'}}
\newcommand \str [1] {\texttt{"#1"}}
\newenvironment{outitemize}{
  \begin{itemize}
  \let \origitem \item \def \item {\origitem[]\hspace{-18pt}}
}{
  \end{itemize}
}
\newcommand \Align [2][t] {
  \begin{array}[#1]{@{}l}
    #2
  \end{array}}

\section{Survol:}
Ce TP vise de creé un shell, qui répresent un interpréteur des commandes destiné
aux systèmes d'exploitation Unix. 

Le ``\texttt{simpleshell}'', c'est l'application conçu par notre echipe, 
va accéder aux fonctionnalités internes du système d'exploitation
et va executer quelques programmes qui sont déjà instaler dans /usr/bin. 

Donc l'utilisateur lance des commandes sous forme d'une entrée texte sur 
la ligne de commande accessible depuis le terminal et notre ``\texttt{simpleshell}''
les va exécutée. 

Le but de notre travail a été de concevoir une implantation en langage C simple, qui
utilise les primitive du système, telles que ``\texttt{exec, fork, dup2, pipe et readdir}''.
Le résultat a été atteint, car on a obtenu une application fiable, tolerante aux fotes et
qui connaître plus des fonctionalités demandés.

En général notre util est capable :
\begin{enumerate}
\item d'exécuter des commandes comme ``\texttt{cat Makefile, ls, pwd, date, man ... }'', 
et la majorité des utils qui sont installer dans le système et que n'exige seulement de 
les demarer.
\item d'exécuter des commandes comme ``\texttt{echo *}''.
\item de rediriger les entrées et sorties des processus: être capable
d'exécuter des commandes comme ``\texttt{cat <Makefile >foo}''.
\item connecter ces processus via des \emph{pipes} pour faire des \emph{pipelines}:
être capable d'exécuter des commandes comme ``\texttt{find -name Makefile | xargs grep ch}''.
\end{enumerate}  


\newpage
\begin{thebibliography}{5}

\bibitem{1}
  The UNIX programming environment, 
  Brian W. Kernighan, Rob Pike,
  1984, Bell Laboratories, New Jersey. 
\bibitem{2}
  Advanced Programming in the UNIX Environment, 
  W. Richard Stevens, Stephen A. Rago,
  Second Edition, 2008  
\bibitem{3}
  Linux Application Development,
  Michael K. Johnson, Eric W. Troan.
  1998.
\bibitem{4}
  GNU/Linux Application Programming, 
  M. Tim Jones,
  First Edition, 2005  
\bibitem{5}
  Latex Companion,
  Frank Mittelbach, Michel Goossens.
  Deusième édition, 2005.  

\end{thebibliography}

Les étapes de ce travail sont les suivantes:
\begin{enumerate}
\item Parfaire sa connaissance de C et Make et POSIX.
\item Lire et comprendre cette donnée.  Cela prendra probablement une partie
  importante du temps total.
\item Lire, trouver, et comprendre les parties importantes du code fourni.
\item Compléter le code fourni.
\item Écrire un rapport.  Il doit décrire \textbf{votre} expérience pendant
  les points précédents: problèmes rencontrés, surprises, choix que vous
  avez dû faire, options que vous avez sciemment rejetées, etc...  Le
  rapport ne doit pas excéder 5 pages.
\end{enumerate}

Ce travail est à faire en groupes de 2 étudiants.  Le rapport, au format
\LaTeX\ exclusivement (compilable sur \texttt{frontal.iro}) et le code sont
à remettre par remise électronique avant la date indiquée.  Aucun retard ne
sera accepté.  Indiquez clairement votre nom au début de chaque fichier.

Si un étudiant préfère travailler seul, libre à lui, mais l'évaluation de
son travail n'en tiendra pas compte.  Si un étudiant ne trouve pas de
partenaire, il doit me contacter au plus vite.  Des groupes de 3 ou plus
sont \textbf{exclus}.

\newpage
\section{Conclusion}

Vous allez devoir implanter une ligne de commande, similaire
à \texttt{/bin/sh}, qui sait:
\begin{enumerate}
\item Démarrer des processus externes: être capable d'exécuter des commandes
  comme ``\texttt{cat Makefile}''.
\item Expansion d'arguments: être capable d'exécuter des commandes comme
  ``\texttt{echo *}''.
\item Rediriger les entrées et sorties de ces processus: être capable
  d'exécuter des commandes comme ``\texttt{cat <Makefile >foo}''.
\item Connecter ces processus via des \emph{pipes} pour faire des
  \emph{pipelines}: être capable d'exécuter des commandes comme
  ``\texttt{find -name Makefile | xargs grep ch}''.
\end{enumerate}



\end{document}
