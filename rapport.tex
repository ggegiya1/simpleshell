\documentclass{article}

\usepackage[utf8]{inputenc}
\usepackage{amsfonts}            %For \leadsto
\usepackage{amsmath}             %For \text

\title{Travail pratique \#1}
\author{Georgiy Gegiya, Cordeleanu Corneliu}

\begin{document}

\maketitle

\newcommand \mML {\ensuremath\mu\textsl{ML}}
\newcommand \kw [1] {\textsf{#1}}
\newcommand \id [1] {\textsl{#1}}
\newcommand \punc [1] {\kw{`#1'}}
\newcommand \str [1] {\texttt{"#1"}}
\newenvironment{outitemize}{
  \begin{itemize}
  \let \origitem \item \def \item {\origitem[]\hspace{-18pt}}
}{
  \end{itemize}
}
\newcommand \Align [2][t] {
  \begin{array}[#1]{@{}l}
    #2
  \end{array}}

\section{Introduction:}

Le terme anglais \texttt{shell} vient à l'origine de la terminologie employée 
avec les premiers systèmes d'exploitation de type Unix où il avait le sens plus 
spécifique de shell Unix. Cette appellation est une métaphore pour désigner la couche
la plus haute des interfaces des systèmes Unix, par opposition à la couche de bas-niveau,
appelée noyau. La \texttt{coque} invoque l'association de l'écran et du clavier,
utilisées par des émulateurs de consoles (les \texttt{coquilles} en analogie 
au chemin complexe d'une architecture client-serveur): nouvelles interfaces systèmes
de la micro-informatique. Avec l'arrivée de la souris et des interfaces graphiques, 
cet anglicisme a fini par être démocratisé pour désigner tous les types d'interfaces
de sortie vidéo d'un système d'exploitation, qu'elles soient textuelles ou graphiques. 
\cite{Widipedia}.

\begin{thebibliography}{2}

\bibitem{Wikidpedia}
  Wikipedia
\bibitem{OtherItem}
  Other Title,
  other bla bla.

\end{thebibliography}

Les étapes de ce travail sont les suivantes:
\begin{enumerate}
\item Parfaire sa connaissance de C et Make et POSIX.
\item Lire et comprendre cette donnée.  Cela prendra probablement une partie
  importante du temps total.
\item Lire, trouver, et comprendre les parties importantes du code fourni.
\item Compléter le code fourni.
\item Écrire un rapport.  Il doit décrire \textbf{votre} expérience pendant
  les points précédents: problèmes rencontrés, surprises, choix que vous
  avez dû faire, options que vous avez sciemment rejetées, etc...  Le
  rapport ne doit pas excéder 5 pages.
\end{enumerate}

Ce travail est à faire en groupes de 2 étudiants.  Le rapport, au format
\LaTeX\ exclusivement (compilable sur \texttt{frontal.iro}) et le code sont
à remettre par remise électronique avant la date indiquée.  Aucun retard ne
sera accepté.  Indiquez clairement votre nom au début de chaque fichier.

Si un étudiant préfère travailler seul, libre à lui, mais l'évaluation de
son travail n'en tiendra pas compte.  Si un étudiant ne trouve pas de
partenaire, il doit me contacter au plus vite.  Des groupes de 3 ou plus
sont \textbf{exclus}.

\newpage
\section{CH: un shell pour les hélvètes}

Vous allez devoir implanter une ligne de commande, similaire
à \texttt{/bin/sh}, qui sait:
\begin{enumerate}
\item Démarrer des processus externes: être capable d'exécuter des commandes
  comme ``\texttt{cat Makefile}''.
\item Expansion d'arguments: être capable d'exécuter des commandes comme
  ``\texttt{echo *}''.
\item Rediriger les entrées et sorties de ces processus: être capable
  d'exécuter des commandes comme ``\texttt{cat <Makefile >foo}''.
\item Connecter ces processus via des \emph{pipes} pour faire des
  \emph{pipelines}: être capable d'exécuter des commandes comme
  ``\texttt{find -name Makefile | xargs grep ch}''.
\end{enumerate}



\end{document}
