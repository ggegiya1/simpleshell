\documentclass{article}

\usepackage[utf8]{inputenc}
\usepackage{amsfonts}            %For \leadsto
\usepackage{amsmath}             %For \text

\title{Travail pratique \#1}
\author{Georgiy Gegiya, Cordeleanu Corneliu}

\begin{document}

\maketitle

\newcommand \mML {\ensuremath\mu\textsl{ML}}
\newcommand \kw [1] {\textsf{#1}}
\newcommand \id [1] {\textsl{#1}}
\newcommand \punc [1] {\kw{`#1'}}
\newcommand \str [1] {\texttt{"#1"}}
\newenvironment{outitemize}{
  \begin{itemize}
  \let \origitem \item \def \item {\origitem[]\hspace{-18pt}}
}{
  \end{itemize}
}
\newcommand \Align [2][t] {
  \begin{array}[#1]{@{}l}
    #2
  \end{array}}

\section{Survol:}
Ce TP vise de creé un shell, qui répresent un interpréteur des commandes destiné
aux systèmes d'exploitation Unix. 
\\\\
Le ``\texttt{simpleshell}'', c'est l'application conçu par notre echipe, 
va accéder aux fonctionnalités internes du système d'exploitation
et va executer quelques programmes qui sont déjà instaler dans /usr/bin. 
\\\\
Le but de notre travail a été de concevoir une implantation en langage C simple, qui
utilise les primitive du système, telles que ``\texttt{exec, fork, dup2, pipe et readdir}''.
\\\\
Le résultat a été atteint, car on a obtenu une application fiable, tolerante aux fotes et
qui connaître plus des fonctionalités demandés.
\\\\
En général notre util est capable :
\begin{enumerate}
\item d'exécuter des commandes comme ``\texttt{cat Makefile, ls, pwd, date, ...}'' 
et la majorité des utils qui sont installer dans le système et que n'exige seulement de 
les demarer.
\item d'exécuter des commandes comme ``\texttt{echo *}''.
\item de rediriger les entrées et sorties des processus: être capable
d'exécuter des commandes comme ``\texttt{cat <Makefile >foo}''.
\item connecter ces processus via des \emph{pipes} pour faire des \emph{pipelines}:
être capable d'exécuter des commandes comme ``\texttt{find -name Makefile | xargs grep ch}''.
\end{enumerate}  

\newpage
\section{Implementation}
Les étapes qu'on a choisi pour implanter le shell sont suivantes:
\begin{enumerate}
\item Donc l'utilisateur lance des commandes sous forme d'une entrée texte sur 
la ligne de commande accessible depuis le terminal et notre ``\texttt{simpleshell}''
les exécutes.
\item A l'aide d'une fonction parseur ``\texttt{parse command}'' et une fonction
``\texttt{trim}'' on analyse la ligne de commande et on crée des commandes dans les
structures de 4 champs: ``\texttt{command, input, output et status}''.
\item Ensuite tous les commandes sont stoker dans un tableu. En fait la création des
commandes c'étais le plus gros travail dans cet dovoir, car tout le reste c'étais 
la compréhension des proccesus et de la création des pipelines.
\item Aprés le stokage, les commandes s'execute une par une. Au debut on va faire un appel
a une primitive ``\texttt{fork}'', pour crée dans le système un nouveu processus. 
Et ensuite on va executer avec une autre primitive ``\texttt{execvp}''.
\item Aprés le stokage, les commandes s'execute une par une. Au debut on va faire un appel
a une primitive ``\texttt{fork}'', pour crée dans le système un nouveu processus. 
Et ensuite on va les executer à l'aide d'une autre primitive ``\texttt{execvp}''.
\item Pour rediriger les entrées et les sorties entres programmes on utilise la primitive
du système ``\texttt{dup2}'', mais avant d'utiliser la fonction ``\texttt{execvp}''.
\item Ensuite pour être sur que le proccesus enfant termine sa execution on s'en serve de
deux marcros ``\texttt{WIFEXITED et WIFSIGNALED}'', crée par la librarie GNU.
\end{enumerate}


\newpage
\section{Conclusion}
\begin{enumerate}
\item La prémiere conlusion qu'on voudra de dire est l'utilisation parfaite de langage C, 
pour crée des application de type CMD. 
\\\\Le fait que le kernel et écrite en C on a la posibilité d'utiliser directement beaucoup 
des primitives du système et cela nous permète de gérérer d'un façon plus efficace des resources,
comme par exemples la mémoire et l'execution plus rapides des applications.
\item En faisant ce TP on a compri meilleux le fonctionnement des processus. En fait on a 
remarquer le fait que l'appel de primitive ``\texttt{fork}'' va résulter de foix l'affichage
des variables du evironnement. Cela veut dire que utilisation de cette primitive est déconseiller,
car elle double la mémoire. 
\\\\
En fait les systèmes modernes utilise des nouvelle techniques, comme par exemple 
``\texttt{copy-on-write}'', dont leur efficatiée est plus élever, car cex types utilises 
le partage des données en mémoire pour les deux types des processus: parent et enfant.
Voir ``\texttt{[2] Section 8.3}''.
\item ...

\end{enumerate}

\newpage
\begin{thebibliography}{5}

\bibitem{1}
  The UNIX programming environment, 
  Brian W. Kernighan, Rob Pike,
  1984, Bell Laboratories, New Jersey. 
\bibitem{2}
  Advanced Programming in the UNIX Environment, 
  W. Richard Stevens, Stephen A. Rago,
  Second Edition, 2008  
\bibitem{3}
  Linux Application Development,
  Michael K. Johnson, Eric W. Troan.
  1998.
\bibitem{4}
  GNU/Linux Application Programming, 
  M. Tim Jones,
  First Edition, 2005  
\bibitem{5}
  Latex Companion,
  Frank Mittelbach, Michel Goossens.
  Deusième édition, 2005.  

\end{thebibliography}

\end{document}
